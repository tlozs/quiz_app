\documentclass{article}
\usepackage[utf8]{inputenc}
\usepackage[
    top=1in,
    bottom=1in,
    left=1in,
    right=1in
]{geometry}
\usepackage{lmodern}
\usepackage[ngerman]{babel}
\usepackage{hyperref}
\usepackage{listingsutf8}
\usepackage{color}
\usepackage{xcolor}
\usepackage{courier}
\usepackage[many]{tcolorbox}

\tcolorboxenvironment{lstlisting}{
  spartan,
  frame empty,
  boxsep=0mm,
  left=1mm,right=1mm,top=-1mm,bottom=-1mm,
  colback=black,
}

\lstset{
    language=C,
    backgroundcolor=\color{black},
    basicstyle=\ttfamily\small\color{white},
    keywordstyle=\bfseries\color{cyan},
    commentstyle=\color{green},
    stringstyle=\color{orange},
    numbers=left,
    numberstyle=\tiny\color{gray},
    stepnumber=1,
    numbersep=5pt,
    showspaces=false,
    showstringspaces=false,
    showtabs=false,
    tabsize=4,
    captionpos=b,
    breaklines=true,
    breakatwhitespace=false,
    escapeinside={\%*}{*)},
}


\title{Entwicklerdokumentation von Quiz App}
\author{Zsolt Gaál}
\date{\today}

\begin{document}

\maketitle

\tableofcontents

\newpage

\section{Einführung}
In jedem Fall enthalten die Header-Dateien die Gebrauchsanweisungen für die einzelnen Funktionen, d.h. eine detaillierte Dokumentation über die Funktionsweise, Parameter und Rückgabewerte.

\section{Dateistruktur}

\subsection{\texttt{main.c}}
Enthält die \texttt{main} Funktion, die das Programm startet und alle Hauptprozesse aufruft:

\begin{lstlisting}
int main(int argc, char *argv[]) {
    srand(time(NULL));  // random seed
    clear_screen();     // Konsole leeren
    init_quiz();        // Speicherplatz allokieren
    
    read_all_input(argc, argv); // Einlesen
    shrink_quiz_size();         // Speicherplatzminimierung
    play_quiz();                // Spiel
    evaluate_quiz();            // Auswertung

    free_quiz();        // Speicherplatz freigeben
    return 0;
}
\end{lstlisting}

\subsection{\texttt{comm.c}}
Dient der Vereinfachung der Kommunikation mit dem Benutzer. Anstelle von \texttt{printf} wird eine eigene Nachrichtenanzeigefunktion mit verschiedenen Nachrichtentypen implementiert. Enthält die Menüanzeige, den Spielmoduswähler und den Tastenanschlagsbeobachter.

\subsection{\texttt{fs\_utils.c}}
Enthält kürzere Hilfsfunktionen: Zeichenkettenmanipulationen, Anforderungsüberprüfungen.

\subsection{\texttt{fs\_read.c}}
Verantwortlich für das Einlesen der Eingabedateien. Navigiert im Dateisystem, benachrichtigt den Benutzer bei Fehlern und fordert gegebenenfalls eine Intervention an.

\noindent Einige globale Konstanten:
\begin{lstlisting}
const char * const input_root = "input";
const char * const allowed_extensions[] = {".txt", ".tsv"};
const int allowed_extensions_size = sizeof(allowed_extensions) / sizeof(allowed_extensions[0]);
\end{lstlisting}

\noindent Das Haupteinlesen:
\begin{lstlisting}
void read_all_input(int argc, char *argv[]) {
    int i;

    // wenn kein Eingabeverzeichnis vorhanden ist, erstellen wir eines
    if (no_input_root()) {
        print_message(ERROR, "Input directory not found.");
        print_message(QUESTION, "Would you like to create the input directory? (y/n) ");
\end{lstlisting}
\begin{lstlisting}

        if (getchar_equals('y') && create_input_root()) {
            print_message(INFO, "Input directory created.");
            print_message(INFO, "Place your input files in the input directory, then press any key to continue.");
            print_message(QUESTION, "If there is no ANY key on your keyboard, consult your local keyboard vendor...");
            getchar_equals(0);
        } 
        else {
            print_message(FATAL, "Could not create input directory.");
        }
        print_message(INFO, "");
    }

    // wenn keine Parameter angegeben sind, lesen wir den gesamten Input-Ordner ein
    if (argc == 1) {
        print_message(WARNING, "No input parameters specified in the command line.");
        print_message(QUESTION, "Would you like to read all files in the input directory? (y/n) ");
        if (getchar_equals('y'))
            try_read_input(path_join(input_root, NULL));
    }
    else // ohne Fehlerbehandlung w%*\color{green}ä*)re es nur so :)
        for (i = 1; i < argc; i++)
            // da wir relativ zum Eingabeverzeichnis suchen, f%*\color{green}ü*)gen wir den gesuchten Pfad mit dem Eingabeverzeichnis zusammen
            try_read_input(path_join(input_root, argv[i]));

    return;
}
\end{lstlisting}
\noindent So versuchen wir, vom angegebenen Pfad zu lesen:
\begin{lstlisting}
        void try_read_input(char *input_path) {
            if (!inside_input_root(input_path)) {   // es wurde gesagt, dass wir aus dem input-Verzeichnis lesen
                print_message(ERROR, "Input '%s' is not within the input directory.", input_path);
            }
            // zuerst versuchen wir, die Eingabe als Datei zu lesen, und falls das nicht klappt, als Ordner
            else if (!try_read_file(input_path) && !try_read_folder(input_path)) {
                print_message(ERROR, "Could not read input path: '%s'", input_path);
            }

            free(input_path);   // Speicherplatz freigeben
            return;
        }
\end{lstlisting}

\subsection{\texttt{quiz.c}}
Verwalten der Quiz-Datenbank und Implementierung der Quiz-Logik. Hier erfolgt die Speicherzuweisung und freigabe, das Hinzufügen neuer Fragen und die Steuerung der Hauptspielschleife.

\newpage
\noindent   Die Quiz-Datenbank:
\begin{lstlisting}
typedef struct Quiz {
    %*\bfseries\color{cyan}QAPair*) **qas;   // die Liste der Frage-Antwort-Paare
    int size;       // die Anzahl der gespeicherten Daten
    int capacity;   // die Gr%*\color{green}öß*)e des f%*\color{green}ü*)r die Liste reservierten Speichers
} Quiz;
\end{lstlisting}

\noindent Der Aufbau der einzelnen Daten ist äußerst einfach:
\begin{lstlisting}
typedef struct QAPair {
    char *question; // die Frage im Speicher
    char *answer;   // die Antwort
} QAPair;
\end{lstlisting}

\noindent Das Hinzufügen neuer Daten zur Datenbank erfolgt wie folgt:
\begin{lstlisting}
int extend_quiz(%*\bfseries\color{cyan}QAPair*) *qa) {
    int success = 0;    // ob es erfolgreich war
    
    // es ist schwierig, nach nichts zu fragen
    if (qa != NULL) {
        // wenn wir mehr Platz brauchen
        if (quiz->size == quiz->capacity) {
            quiz->capacity *= 2;
            
            quiz->qas = realloc(quiz->qas, quiz->capacity * sizeof(QAPair*));
            if(quiz->qas == NULL)
                print_message(FATAL, "Memory allocation failed.");
        }

        // wir haben schon genug Platz
        quiz->qas[quiz->size++] = qa;
        success = 1;
    }

    return success;
}
\end{lstlisting}

\noindent Die Hauptspielschleife:
\begin{lstlisting}
void play_quiz() {
    int aktuelle;    // der Index der zuf%*\color{green}ä*)llig ausgew  %*\color{green}ä*)hlten Frage
    int exit = 0;   // ob eine Beendigungsanforderung eingegangen ist
    int range = quiz->size;
    if (range == 0)
        print_message(FATAL, "Could not read any data from the specified inputs.");

    gamemode_select();  // Auswahl des Spielmodus
    start_timer();      // Uhr startet

    // solange wir noch Fragen haben oder keine Beendigungsanforderung eingegangen ist
    while (range && !exit) {
        %*\bfseries\color{cyan}QAPair*) *qa = random_question(range, &current);  // wir w%*\color{green}ä*)hlen eine Frage aus
        exit = ask_and_correct_question(qa);            // Frage stellen und Antwort korrigieren
        swap_qas(current, --range);         // wir tauschen mit der letzte Element
\end{lstlisting}
\begin{lstlisting}
        // Im Endloserlebnis, wenn keine Fragen mehr %*\color{green}ü*)brig sind, werden alle Fragen wieder in den Pool zur%*\color{green}ü*)ckgelegt
        if ((gamemode == INFINITE || gamemode == INFINITE_REVERSED) && range == 0)
            range = quiz->size;            
    }

    stop_timer();       // Uhr anhalten
    
    return;
}
\end{lstlisting}

\subsection{\texttt{evaluation.c}}
Verantwortlich für die Nachverfolgung und Bewertung der Benutzerleistung.

\noindent Hier wird die Frage gestellt und die Antwort korrigiert:
\begin{lstlisting}
int ask_and_correct_question(QAPair *qa) {
    char user_answer[ANSWER_SIZE];  // hier wird die Antwort eingelesen
    char* question;         // das werden wir fragen
    char* correct_answer;   // die richtige Antwort
    int exit = 0;           // ob eine Beendigungsanforderung eingegangen ist

    // Im umgekehrten Spielmodus vertauschen wir die Frage und die Antwort
    if (gamemode == ONEROUNDER_REVERSED || gamemode == INFINITE_REVERSED) {
        question = qa->answer;
        correct_answer = qa->question;
    }
    else {
        question = qa->question;
        correct_answer = qa->answer;
    }

    // f%*\color{green}ü*)r die Ergebnisanzeige erforderlich (global)
    asked_questions++;

    // wir stellen die Frage und lesen die Antwort ein
    print_message(INFO, "%s ", question);
    fgets(user_answer, sizeof(user_answer), stdin);

    // wir entfernen das Enter-Zeichen am Ende
    user_answer[strcspn(user_answer, "\n")] = '\0';

    // wir erfassen die Beendigungsanforderung
    if (strcmp(user_answer, "!exit") == 0)
        exit = 1;
    // ob die Antwort richtig war
    else if (strcmp(user_answer, correct_answer) == 0)
        correct_answers++;
    // wenn nicht, h%*\color{green}ä*)tte dies geschrieben werden sollen
    else
        print_message(INCORRECT, "%s", correct_answer);

    // %*\color{green}ä*)sthetischer Zeilenumbruch
    print_message(INFO, "");
    
    return exit;
}
\end{lstlisting}

\end{document}